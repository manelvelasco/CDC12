\section{Ejemplo motivador}


El ejemplo motivador se centra en la descripci�n de un sistema de una sola dimensi�n. Al aplicar el retraso el sistema se puede modelar de las siguientes maneras:
\begin{itemize}
\item Ampliando el espacio de estados con una nueva variable
\item Dejando el sistema sin ampliar y representarlo con dos sistemas independientes
\end{itemize}


\marginpar{\small{Pau, hi ha dos exemples, escull el que m�s t'agradi}}



%***************************************************************************************************************************
%***************************************************************************************************************************
%***************************************************************************************************************************
%***************************************************************************************************************************
%*************EJEMPO UNO****************************************************************************************************
%***************************************************************************************************************************
%***************************************************************************************************************************
%***************************************************************************************************************************
%***************************************************************************************************************************
\subsection{EJEMPLO DIMENSI�N 1}



%***************************************************************************************************************************
%****************MODELO*****************************************************************************************************
%***************************************************************************************************************************



El modelo que queremos controlar est� descrito por las siguinetes ecuaciones continuas
\begin{equation}\label{eq:ejemplo_sistema_conmtinuo}
\dot{x}(t)=ax(t)+bu(t)
\end{equation}

%***************************************************************************************************************************
%*****************MODELO DISCRETO*******************************************************************************************
%***************************************************************************************************************************



El modelo discreto tiene un muestreo regular cada $h$ segundos y una actuaci�n regular $h$ segundos, pero el muestreo y la actuaci�n est�n desfasadaos uno respecto del otro por un tiempo de $\tau$ segundos

Si discretizamos el sistema (\ref{eq:ejemplo_sistema_conmtinuo}) en los puntos en los que se toman las muestras y mantenemos contante la se�al de control entre las actuaciones las ecuaciones que gobiernan la din�mica en los puntos de muestra son 

\begin{equation}\label{eq:ejemplo_modelo_discreto}
	x_{k+1}=e^{ah}+e^{a(h-\tau)}\int_0^{\tau}e^{as}dsbu_{k-1}+\int_0^{h-\tau}e^{as}dsbu_k
\end{equation}


%***************************************************************************************************************************
%*****************MODELO AMPIADO********************************************************************************************
%***************************************************************************************************************************


Si escogemos ampliar el sistema el modelo se puede representar de una manera m�s compacta 

\begin{equation}
	\left[
	\begin{matrix}
		x_{k+1}\\
		z_{k+1}
	\end{matrix}
	\right]=
	\left[
	\begin{matrix}
		e^{ah} & e^{a(h-\tau)}\int_0^{\tau}e^{as}ds\\
		0 & 0
	\end{matrix}
	\right]
	\left[\begin{matrix}
		x_{k}\\
		z_{k}
	\end{matrix}\right]
	+\left[\begin{matrix}
		\int_0^{h-\tau}e^{as}dsb\\
		1
	\end{matrix}\right]u_k
\end{equation}

%***************************************************************************************************************************
%******************MODELO AMPLIADO CON VALORES NUMERICOS********************************************************************
%***************************************************************************************************************************



Para el caso que nos ocupa tomaremos $a=1$, $b=1$, $h=1$ y $\tau=0.5$ 
que deja el sistema 
\begin{equation}
	\left[
	\begin{matrix}
		x_{k+1}\\
		z_{k+1}
	\end{matrix}
	\right]=
	\left[\begin{array}{cc} 2.718 & 1.07\\ 0 & 0 \end{array}\right]
	\left[\begin{matrix}
		x_{k}\\
		z_{k}
	\end{matrix}\right]
	+\left[\begin{array}{c} 
		0.6487\\ 
		1.0 
	\end{array}\right]
	u_k
\end{equation}

%***************************************************************************************************************************
%****************DISE�O DEL CONTROLADOR*************************************************************************************
%***************************************************************************************************************************

Una vez obtenido el modelo es posible dise�ar un controlador ($u_k=Lx_k$) que situe los polos del sistema en lazo cerrado en el lugar deseado. En nuestro caso, siendo un sistema discreto, escogemos los polos en $0.7104 \pm 1.2784i$. 


%***************************************************************************************************************************
%****************SISTEMA EN LAZO CERRADO************************************************************************************
%***************************************************************************************************************************

El controlador que situa los polos en el lugar deseado es $L=\left[\begin{array}{cc} - 2 &  0 \end{array}\right]$
Y el sistema en lazo cerrado queda representado por
\begin{equation}\label{eq:ejemplo_estado_ampliado_lazo_cerrado}
	\left[
	\begin{matrix}
		x_{k+1}\\
		z_{k+1}
	\end{matrix}
	\right]=
	\left[\begin{array}{cc} 1.421 & 1.07\\ -2.0 & 0 \end{array}\right]
	\left[\begin{matrix}
		x_{k}\\
		z_{k}
	\end{matrix}\right]
\end{equation}

%***************************************************************************************************************************
%************REPRESENTACION ALTERNATIVA*************************************************************************************
%***************************************************************************************************************************

Una manera alternativa de representar el sistema consiste en no ampliar el sistestema pare representarlo, si sustituimos todos los valores en la ecuaci�n (\ref{eq:ejemplo_modelo_discreto}) obtenemos

\begin{equation}
	x_{k+1}=2.718x_k+1.07u_{k-1}+0.6487u_k
\end{equation}
Y, teniendo en cuenta que $u_k=-2x_k$
\begin{equation}
	x_{k+1}=2.718x_k-2.139x_{k-1}-1.297x_k
\end{equation}
que simplificando
\begin{equation}\label{eq:ejemplo_modelo_discreto_preparado}
	x_{k+1}=1.421x_k-2.139x_{k-1}
\end{equation}

%***************************************************************************************************************************
%*************MATRIZ EN LAZO CERRADO****************************************************************************************
%***************************************************************************************************************************

Llegados a este punto supongamos que existe un valor $P$ que es capaz de representar el sistema en lazo cerrado, de manera que $x_{k+1}=Px_k$. Si tomamos la din�mca a la inversa obtenemos que $x_{k}=P^{-1}x_{k+1}$ o lo que es lo mismo, $x_{k-1}=P^{-1}x_{k}$. Si sustituimos esta expresi�n en la ecuaci�n (\ref{ejemplo_modelo_discreto_preparado})
\begin{equation}
	x_{k+1}=1.421x_k-2.139P^{-1}x_k
\end{equation}
y simplificamos
\begin{equation}\label{eq:ejemplo_modelo_discreto_listo}
	x_{k+1}=\left[1.421-2.139P^{-1}\right]x_k
\end{equation}
Obtemos la ecuaci�n que representa la din�mica en lazo cerrado, que es precisamente la que hemos supuesto que exist�a. De lo anterior se deduce que
\begin{equation}
	P=1.421-2.139P^{-1}
\end{equation}
reordenando
\begin{equation}\label{eq:ejemplo_modelo_discreto_resultado}
	P^2=1.421P-2.139
\end{equation}
De donde es posible obtener que $P=0.7104 \pm 1.2784i$

Es decir, el sistema puede representarse de dos maneras diferentes 
\begin{equation}
	x_{k+1}=(0.7104 + 1.2784i)x_k
\end{equation}
y
\begin{equation}
	x_{k+1}=(0.7104 - 1.2784i)x_k
\end{equation}

Ambas son soluci�n de la ecuacion (\ref{eq:ejemplo_modelo_discreto_resultado}), y cualquiera de las dos sirve para representar la din�mica del sistema.

El modelo (\ref{eq:ejemplo_estado_ampliado_lazo_cerrado}) tiene dos valores propios, y los modelos que hemos deducido tienen cada uno de ellos un valor propio del sistema ampliado. Utilizando uno de ellos es posible recontruir de forma completa el estado del sistema ampliado



%***************************************************************************************************************************
%***************************************************************************************************************************
%***************************************************************************************************************************
%***************************************************************************************************************************
%*************EJEMPO DOS****************************************************************************************************
%***************************************************************************************************************************
%***************************************************************************************************************************
%***************************************************************************************************************************
%***************************************************************************************************************************





\subsection{EJEMPO DIMENSI�N 2}
\marginpar{\small{Encara estic retocant aquest exemple}}
La ecuaci�n de la din�mica de un integrador doble est� descrita por
\begin{eqnarray}
\left[\begin{array}{c} \dot{x}_1 \\
\dot{x}_2
\end{array}\right]=\left[\begin{array}{cc} 0 & 1 \\
   0 & 0
\end{array}\right]\left[\begin{array}{c} x_1 \\
x_2
\end{array}\right]+ \left[\begin{array}{c} 0 \\
1
\end{array}\right]u
\end{eqnarray}

Si se realiza un control discreto del sistema con periodo $h$ y con
un retraso en la aplicaci�n del control de $\tau$ el sistema queda
representado por

\begin{eqnarray}
\left[\begin{array}{c} x_1 \\
x_2
\end{array}\right]_{k+1}=\left[\begin{array}{cc} 1 & h \\
   0 & 1
\end{array}\right]\left[\begin{array}{c} x_1 \\
x_2
\end{array}\right]_k+ \left[\begin{array}{c} \frac{(h-\tau)^2}{2} \\
h-\tau
\end{array}\right]u_k+\left[\begin{array}{c} \tau h- \frac{\tau^2}{2} \\
\tau
\end{array}\right]u_{k-1}
\end{eqnarray}

Si suponemos que $u_k=\left[\begin{array}{cc} l_1 & l_2
\end{array}\right]\left[\begin{array}{c} x_1 \\
x_2
\end{array}\right]_k$ entonces obtenemos

\begin{eqnarray}
\left[\begin{array}{c} x_1 \\
x_2
\end{array}\right]_{k+1}=\left(\left[\begin{array}{cc} 1 & h \\
   0 & 1
\end{array}\right]+ \left[\begin{array}{c} \frac{(h-\tau)^2}{2} \\
h-\tau
\end{array}\right]\left[\begin{array}{cc} l_1 & l_2
\end{array}\right]\right)\left[\begin{array}{c} x_1 \\
x_2
\end{array}\right]_k+\left[\begin{array}{c} \tau h- \frac{\tau^2}{2} \\
\tau
\end{array}\right]\left[\begin{array}{cc} l_1 & l_2
\end{array}\right]\left[\begin{array}{c} x_1 \\
x_2
\end{array}\right]_{k-1}
\end{eqnarray}

Y suponiendo que existe $\hat{\Phi}_{cl}$ el sistema queda
representado por
\begin{eqnarray}
\left[\begin{array}{c} x_1 \\
x_2
\end{array}\right]_{k+1}=\left(\left[\begin{array}{cc} 1 & h \\
   0 & 1
\end{array}\right]+ \left[\begin{array}{c} \frac{(h-\tau)^2}{2} \\
h-\tau
\end{array}\right]\left[\begin{array}{cc} l_1 & l_2
\end{array}\right]+\left[\begin{array}{c} \tau h- \frac{\tau^2}{2} \\
\tau
\end{array}\right]\left[\begin{array}{cc} l_1 & l_2
\end{array}\right]\hat{\Phi}_{cl}^{-1}\right)\left[\begin{array}{c} x_1 \\
x_2
\end{array}\right]_{k}
\end{eqnarray}
Que da como ecuaci�n a resolver
\begin{eqnarray}
\hat{\Phi}_{cl}^2-\left(\left[\begin{array}{cc} 1 & h \\
   0 & 1
\end{array}\right]+ \left[\begin{array}{c} \frac{(h-\tau)^2}{2} \\
h-\tau
\end{array}\right]\left[\begin{array}{cc} l_1 & l_2
\end{array}\right]\right)\hat{\Phi}_{cl}-\left[\begin{array}{c} \tau h- \frac{\tau^2}{2} \\
\tau
\end{array}\right]\left[\begin{array}{cc} l_1 & l_2
\end{array}\right]=0
\end{eqnarray}

Si dise�amos un controlador para el sistema sin
retardos, y tomando $h=1$ vemos que $l_1=-1$ y $l_2=-\frac{3}{2}$
quedando el sistema representado por
\[
\hat{\Phi}_{cl}=\left[ \begin{array}{cc}  1.5 & 1.75
\\\noalign{\medskip} 1& 2.5\end{array} \right]
\]
lo que situa el sistema con los polos en $\lambda_{1,2}=0$.

Introduciendo un retardo de $0.1s$ al sistema, las soluciones son
\[
\hat{\Phi}_{cl1}=  \left[ \begin {array}{cc}  
-0.5892&  
-0.8838\\\noalign{\medskip}
0.6211&
0.9314\end {array} \right]
\]

\[
\hat{\Phi}_{cl2}=  \left[ \begin {array}{cc}  
0.2671+0.0089i&  
0.4005+0.0134i\\\noalign{\medskip}
-0.2104+0.4268i&
-0.3157+0.6401i\end {array} \right]
\]

\[
\hat{\Phi}_{cl3}=  \left[ \begin {array}{cc}  
0.2671-0.0089i&  
0.4005-0.0134i\\\noalign{\medskip}
-0.2104-0.4268i&
-0.3157-0.6401i\end {array} \right]
\]

\[
\hat{\Phi}_{cl4}=  \left[ \begin {array}{cc}  
0.2448&  
0.4013\\\noalign{\medskip}
-1.269&
-0.342\end {array} \right]
\] 

\[
\hat{\Phi}_{cl5}=  \left[ \begin {array}{cc}  
 0.7000-0.3247i&  
0.3393-0.3075i\\\noalign{\medskip}
 -0.7895-0.3418i&
 -0.4063-0.3246i\end {array} \right]
\] 
 
\[
\hat{\Phi}_{cl6}=  \left[ \begin {array}{cc}  
 0.7000+0.3247i&  
0.3393+0.3075i\\\noalign{\medskip}
 -0.7895+0.3418i&
 -0.4063+0.3246i\end {array} \right]
\] 

Mientras que la matriz ampliada est� representada por 
\[
\hat{\Phi}_{cl6}=  \left[ \begin {array}{ccc}  
 1.4050   & 1.6075 &   0.0950\\\noalign{\medskip}
    0.9000 &   2.3500&    0.1000\\\noalign{\medskip}
    1.0000  &  1.5000 &        0
\end {array} \right]
\] 

