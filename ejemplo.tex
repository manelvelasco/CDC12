\section{Ejemplo motivador}
El ejemplo motivador se centra en la descripci�n de un sistema de una sola dimensi�n. Al aplicar el retraso el sistema se puede modelar de las siguientes maneras:
\begin{itemize}
\item Ampliando el espacio de estados con una nueva variable
\item Dejanso el sistema sin ampliar y representarlo con dos sistemas independientes
\end{itemize}


%***************************************************************************************************************************
%****************MODELO*****************************************************************************************************
%***************************************************************************************************************************



El modelo que queremos controlar est� descrito por las siguinetes ecuaciones continuas
\begin{equation}\label{eq:ejemplo_sistema_conmtinuo}
\dot{x}(t)=ax(t)+bu(t)
\end{equation}

%***************************************************************************************************************************
%*****************MODELO DISCRETO*******************************************************************************************
%***************************************************************************************************************************



El modelo discreto tiene un muestreo regular cada $h$ segundos y una actuaci�n regular $h$ segundos, pero el muestreo y la actuaci�n est�n desfasadaos uno respecto del otro por un tiempo de $\tau$ segundos

Si discretizamos el sistema (\ref{eq:ejemplo_sistema_conmtinuo}) en los puntos en los que se toman las muestras y mantenemos contante la se�al de control entre las actuaciones las ecuaciones que gobiernan la din�mica en los puntos de muestra son 

\begin{equation}\label{eq:ejemplo_modelo_discreto}
	x_{k+1}=e^{ah}+e^{a(h-\tau)}\int_0^{\tau}e^{as}dsbu_{k-1}+\int_0^{h-\tau}e^{as}dsbu_k
\end{equation}


%***************************************************************************************************************************
%*****************MODELO AMPIADO********************************************************************************************
%***************************************************************************************************************************


Si escogemos ampliar el sistema el modelo se puede representar de una manera m�s compacta 

\begin{equation}
	\left[
	\begin{matrix}
		x_{k+1}\\
		z_{k+1}
	\end{matrix}
	\right]=
	\left[
	\begin{matrix}
		e^{ah} & e^{a(h-\tau)}\int_0^{\tau}e^{as}ds\\
		0 & 0
	\end{matrix}
	\right]
	\left[\begin{matrix}
		x_{k}\\
		z_{k}
	\end{matrix}\right]
	+\left[\begin{matrix}
		\int_0^{h-\tau}e^{as}dsb\\
		1
	\end{matrix}\right]u_k
\end{equation}

%***************************************************************************************************************************
%******************MODELO AMPLIADO CON VALORES NUMERICOS********************************************************************
%***************************************************************************************************************************



Para el caso que nos ocupa tomaremos $a=1$, $b=1$, $h=1$ y $\tau=0.5$ 
que deja el sistema 
\begin{equation}
	\left[
	\begin{matrix}
		x_{k+1}\\
		z_{k+1}
	\end{matrix}
	\right]=
	\left[\begin{array}{cc} 2.718 & 1.07\\ 0 & 0 \end{array}\right]
	\left[\begin{matrix}
		x_{k}\\
		z_{k}
	\end{matrix}\right]
	+\left[\begin{array}{c} 
		0.6487\\ 
		1.0 
	\end{array}\right]
	u_k
\end{equation}

%***************************************************************************************************************************
%****************DISE�O DEL CONTROLADOR*************************************************************************************
%***************************************************************************************************************************

Una vez obtenido el modelo es posible dise�ar un controlador ($u_k=Lx_k$) que situe los polos del sistema en lazo cerrado en el lugar deseado. En nuestro caso, siendo un sistema discreto, escogemos los polos en $0.7104 \pm 1.2784i$. 


%***************************************************************************************************************************
%****************SISTEMA EN LAZO CERRADO************************************************************************************
%***************************************************************************************************************************

El controlador que situa los polos en el lugar deseado es $L=\left[\begin{array}{cc} - 2 &  0 \end{array}\right]$
Y el sistema en lazo cerrado queda representado por
\begin{equation}
	\left[
	\begin{matrix}
		x_{k+1}\\
		z_{k+1}
	\end{matrix}
	\right]=
	\left[\begin{array}{cc} 1.421 & 1.07\\ -2.0 & 0 \end{array}\right]
	\left[\begin{matrix}
		x_{k}\\
		z_{k}
	\end{matrix}\right]
\end{equation}

%***************************************************************************************************************************
%************REPRESENTACION ALTERNATIVA*************************************************************************************
%***************************************************************************************************************************

Una manera alternativa de representar el sistema consiste en no ampliar el sistestema pare representarlo, si sustituimos todos los valores en la ecuaci�n (\ref{eq:ejemplo_modelo_discreto}) obtenemos

\begin{equation}
	x_{k+1}=2.718x_k+1.07u_{k-1}+0.6487u_k
\end{equation}
Y, teniendo en cuenta que $u_k=-2x_k$
\begin{equation}
	x_{k+1}=2.718x_k-2.139x_{k-1}-1.297x_k
\end{equation}
que simplificando
\begin{equation}\label{eq:ejemplo_modelo_discreto_preparado}
	x_{k+1}=1.421x_k-2.139x_{k-1}
\end{equation}

%***************************************************************************************************************************
%*************MATRIZ EN LAZO CERRADO****************************************************************************************
%***************************************************************************************************************************

Llegados a este punto supongamos que existe un valor $P$ que es capaz de representar el sistema en lazo cerrado, de manera que $x_{k+1}=Px_k$. Si tomamos la din�mca a la inversa obtenemos que $x_{k}=P^{-1}x_{k+1}$ o lo que es lo mismo, $x_{k-1}=P^{-1}x_{k}$. Si sustituimos esta expresi�n en la ecuaci�n (\ref{ejemplo_modelo_discreto_preparado})
\begin{equation}
	x_{k+1}=1.421x_k-2.139P^{-1}x_k
\end{equation}
y simplificamos
\begin{equation}\label{eq:ejemplo_modelo_discreto_listo}
	x_{k+1}=\left[1.421-2.139P^{-1}\right]x_k
\end{equation}
Obtemos la ecuaci�n que representa la din�mica en lazo cerrado, que es precisamente la que hemos supuesto que exist�a. De lo anterior se deduce que
\begin{equation}\label{eq:ejemplo_modelo_discreto_resultado}
	P=1.421-2.139P^{-1}
\end{equation}
reordenando
\begin{equation}\label{eq:ejemplo_modelo_discreto_resultado}
	P^2=1.421P-2.139
\end{equation}
De donde es posible obtener que $P=0.7104 \pm 1.2784i$

Es decir, el sistema puede representarse de dos maneras diferentes 
\begin{equation}
	x_{k+1}=(0.7104 + 1.2784i)x_k
\end{equation}
y
\begin{equation}
	x_{k+1}=(0.7104 - 1.2784i)x_k
\end{equation}


